\documentclass[12pt,oneside,letterpaper]{article}
\usepackage{amsmath, amsthm, amssymb, amsfonts}
\usepackage{braket}
\usepackage{appendix}
\usepackage{mathrsfs,mathtools}
\usepackage{graphicx,wrapfig,graphics,lipsum}
\usepackage{cite}
\usepackage{csquotes}
\hyphenation{thatshouldnot}
\usepackage[margin=1in]{geometry}
\usepackage{stmaryrd}
\usepackage{color}
\usepackage{xcolor}

\begin{document}
	All of my points have been satisfactorily addressed. I do have one minor and highly subjective comment regarding the revisions. After Eqn. 5, there is added text about why this is an efficiency measure. To me, this seems verbose and somewhat obvious. In my first review, I had a question about what was Eqn. 11:
	\begin{equation*}
		E_1 = \frac{\langle r_1 \rangle}{\delta_1 \langle r_A\rangle + \beta_1 \langle x_1 \rangle},
	\end{equation*}
	with an analogous expression for $E_2$. All of my concerns that I raised in the first point were resolved when the authors said these equations had typos and replaced them with the correct expression. If the explanation after Eqn. 5 is there just to elaborate on this and resolve my first comment, I don't think it needs to be there.
	
	I accept their explanation that the point of the demonstrations provided in the paper was to show the existence of a set of parameters that avoids diverging fluctuations, though I do appreciate their further demonstration that this is a more robust feature of the model. Likewise, I am happy with the added biological context on different assembly mechanisms.
\end{document}