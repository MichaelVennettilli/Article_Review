\documentclass[12pt,oneside,letterpaper]{article}
\usepackage{amsmath, amsthm, amssymb, amsfonts}
\usepackage{braket}
\usepackage{appendix}
\usepackage{mathrsfs,mathtools}
\usepackage{graphicx,wrapfig,graphics,lipsum}
\usepackage{cite}
\usepackage{csquotes}
\hyphenation{thatshouldnot}
\usepackage[margin=1in]{geometry}
\usepackage{stmaryrd}
\usepackage{color}
\usepackage{xcolor}

\begin{document}

Motivated by experimentally observed quasi-one-dimensional structures, the authors study size regulation of such structures. The key ingredients are breaking of cell-cell junctions and formation of new junctions through division. The breaking occurs due to cell activity. The first part of the paper treats this as an active, persistent particle model. They obtain an expression of the breaking rate using Kramer's theory. This is used as input for the next step, which adds in cell division. Analytical methods, stochastic simulation, and numerics are used to characterize the distribution of chain lengths and the coefficient of variation. The two variations they consider are whether growth occurs uniformly or strictly at the boundary and whether the chain needs to be sufficiently large too break. They find that localized growth at the boundaries and breaking localized to the middle of the chain both reduce the coefficient of variation.

\end{document}